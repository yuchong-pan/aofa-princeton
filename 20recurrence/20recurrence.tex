
\documentclass[letterpaper, reqno,11pt]{article}
\usepackage[margin=1.0in]{geometry}
\usepackage{color,latexsym,amsmath,amssymb}
\usepackage{fancyhdr}
\usepackage{amsthm}
\usepackage{graphicx}

\newdimen\R
\R=0.8cm
\pagestyle{fancy}
\lhead{Analysis of Algorithms Exercises 2}
\rhead{Yuchong Pan}
\begin{document}
\pagenumbering{arabic}
\title{Analysis of Algorithms Exercises 2}
\author{Yuchong Pan}
\date{\today}
\newtheorem{thm}{Theorem}[section]
\newtheorem{lemma}{Lemma}[section]
\newtheorem{defn}{Definition}[section]
\maketitle
%

\noindent {\bf Exercise 2.17} Simplifying the right-hand side, we have
\begin{equation*}
    A_N=\frac{N-6}{N}A_{N-1}+2
\end{equation*}
Multiplying both sides by $N(N-1)\cdots (N-5)$,
\begin{equation*}
    N(N-1)\cdots (N-5)A_N=(N-1)(N-2)\cdots (N-6)A_{N-1}+2N(N-1)\cdots (N-5)
\end{equation*}
Iterating, we have
\begin{equation*}
    N(N-1)\cdots (N-5)A_N=2\sum_{6\leq j\leq N}j(j-1)\cdots (j-5)
\end{equation*}
Simplifying, we have
\begin{equation*}
    A_N=\frac{2}{N(N-1)\cdots (N-5)}\sum_{6\leq j\leq N}j(j-1)\cdots (j-5)
\end{equation*}

\end{document}
